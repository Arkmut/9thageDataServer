
%%% Input parameters %%%

\define@key{unit}{subentrynumber}{\def\unit@subentrynumber{#1}}
\define@key{unit}{name}{\def\unit@name{#1}}
\define@key{unit}{QRSname}{\def\unit@QRSname{#1}}
\define@key{unit}{logo}{\def\unit@logo{#1}}
\define@key{unit}{speciallogo}{\def\unit@speciallogo{#1}}%  *** obsolete
\define@key{unit}{secondlogo}{\def\unit@secondlogo{#1}}
\define@key{unit}{optionallogo}{\def\unit@optionallogo{#1}}
\define@key{unit}{secondoptionallogo}{\def\unit@secondoptionallogo{#1}}
\define@key{unit}{thirdoptionallogo}{\def\unit@thirdoptionallogo{#1}}

\define@key{unit}{cost}{\def\unit@cost{#1}}
\define@key{unit}{unitsize}{\def\unit@unitsize{#1}}
\define@key{unit}{costpermodel}{\def\unit@costpermodel{#1}}
\define@key{unit}{maxunitsize}{\def\unit@maxunitsize{#1}}
\define@key{unit}{maxunitsperarmy}{\def\unit@maxunitsperarmy{#1}}
\define@key{unit}{maxmodelsperarmy}{\def\unit@maxmodelsperarmy{#1}}
\define@key{unit}{maxmountsperarmy}{\def\unit@maxmountsperarmy{#1}}
\define@key{unit}{addrestriction}{\def\unit@addrestriction{#1}}
\define@key{unit}{type}{\def\unit@type{#1}}
\define@key{unit}{size}{\def\unit@size{#1}}
\define@key{unit}{basesize}{\def\unit@basesize{#1}}
\define@key{unit}{basesizenotag}{\def\unit@basesizenotag{#1}}
\define@key{unit}{scoring}{\def\unit@scoring{#1}}
\define@key{unit}{categorynote}{\def\unit@categorynote{#1}}
\define@key{unit}{costnote}{\def\unit@costnote{#1}}
\define@key{unit}{fluff}{\def\unit@fluff{#1}}

\define@key{unit}{globalname}{\def\unit@globalname{#1}}
\define@key{unit}{global@Ad}{\def\unit@global@Ad{#1}}
\define@key{unit}{global@Adfly}{\def\unit@global@Adfly{#1}}
\define@key{unit}{global@Ma}{\def\unit@global@Ma{#1}}
\define@key{unit}{global@Mafly}{\def\unit@global@Mafly{#1}}
\define@key{unit}{global@Di}{\def\unit@global@Di{#1}}
\define@key{unit}{global@Rsr}{\def\unit@global@Rsr{#1}}
\define@key{unit}{globalrules}{\def\unit@globalrules{#1}}

\define@key{unit}{globalnameB}{\def\unit@globalnameB{#1}}
\define@key{unit}{globalB@Ad}{\def\unit@globalB@Ad{#1}}
\define@key{unit}{globalB@Adfly}{\def\unit@globalB@Adfly{#1}}
\define@key{unit}{globalB@Ma}{\def\unit@globalB@Ma{#1}}
\define@key{unit}{globalB@Mafly}{\def\unit@globalB@Mafly{#1}}
\define@key{unit}{globalB@Di}{\def\unit@globalB@Di{#1}}
\define@key{unit}{globalB@Rsr}{\def\unit@globalB@Rsr{#1}}
\define@key{unit}{globalrulesB}{\def\unit@globalrulesB{#1}}

\define@key{unit}{globalnameC}{\def\unit@globalnameC{#1}}
\define@key{unit}{globalC@Ad}{\def\unit@globalC@Ad{#1}}
\define@key{unit}{globalC@Adfly}{\def\unit@globalC@Adfly{#1}}
\define@key{unit}{globalC@Ma}{\def\unit@globalC@Ma{#1}}
\define@key{unit}{globalC@Mafly}{\def\unit@globalC@Mafly{#1}}
\define@key{unit}{globalC@Di}{\def\unit@globalC@Di{#1}}
\define@key{unit}{globalC@Rsr}{\def\unit@globalC@Rsr{#1}}
\define@key{unit}{globalrulesC}{\def\unit@globalrulesC{#1}}

\define@key{unit}{defensename}{\def\unit@defensename{#1}}
\define@key{unit}{defense@HP}{\def\unit@defense@HP{#1}}
\define@key{unit}{defense@Df}{\def\unit@defense@Df{#1}}
\define@key{unit}{defense@Re}{\def\unit@defense@Re{#1}}
\define@key{unit}{defense@Arm}{\def\unit@defense@Arm{#1}}
\define@key{unit}{defense@Aeg}{\def\unit@defense@Aeg{#1}}
\define@key{unit}{printAeg}{\def\unit@printAeg{#1}}
\define@key{unit}{defenserules}{\def\unit@defenserules{#1}}
\define@key{unit}{defensearmour}{\def\unit@defensearmour{#1}}

\define@key{unit}{defensenameB}{\def\unit@defensenameB{#1}}
\define@key{unit}{defenseB@HP}{\def\unit@defenseB@HP{#1}}
\define@key{unit}{defenseB@Df}{\def\unit@defenseB@Df{#1}}
\define@key{unit}{defenseB@Re}{\def\unit@defenseB@Re{#1}}
\define@key{unit}{defenseB@Arm}{\def\unit@defenseB@Arm{#1}}
\define@key{unit}{defenseB@Aeg}{\def\unit@defenseB@Aeg{#1}}
\define@key{unit}{printAegB}{\def\unit@printAegB{#1}}
\define@key{unit}{defenserulesB}{\def\unit@defenserulesB{#1}}
\define@key{unit}{defensearmourB}{\def\unit@defensearmourB{#1}}

\define@key{unit}{defensenameC}{\def\unit@defensenameC{#1}}
\define@key{unit}{defenseC@HP}{\def\unit@defenseC@HP{#1}}
\define@key{unit}{defenseC@Df}{\def\unit@defenseC@Df{#1}}
\define@key{unit}{defenseC@Re}{\def\unit@defenseC@Re{#1}}
\define@key{unit}{defenseC@Arm}{\def\unit@defenseC@Arm{#1}}
\define@key{unit}{defenseC@Aeg}{\def\unit@defenseC@Aeg{#1}}
\define@key{unit}{printAegC}{\def\unit@printAegC{#1}}
\define@key{unit}{defenserulesC}{\def\unit@defenserulesC{#1}}
\define@key{unit}{defensearmourC}{\def\unit@defensearmourC{#1}}

\define@key{unit}{offensename}{\def\unit@offensename{#1}}
\define@key{unit}{forceoffensenameprint}{\def\unit@forceoffensenameprint{#1}}
\define@key{unit}{offense@Ag}{\def\unit@offense@Ag{#1}}
\define@key{unit}{offense@At}{\def\unit@offense@At{#1}}
\define@key{unit}{offense@Of}{\def\unit@offense@Of{#1}}
\define@key{unit}{offense@St}{\def\unit@offense@St{#1}}
\define@key{unit}{offense@AP}{\def\unit@offense@AP{#1}}
\define@key{unit}{offenserules}{\def\unit@offenserules{#1}}
\define@key{unit}{offenseweapons}{\def\unit@offenseweapons{#1}}
\define@key{unit}{offensecomment}{\def\unit@offensecomment{#1}}

\define@key{unit}{offensenameB}{\def\unit@offensenameB{#1}}
\define@key{unit}{offenseB@Ag}{\def\unit@offenseB@Ag{#1}}
\define@key{unit}{offenseB@At}{\def\unit@offenseB@At{#1}}
\define@key{unit}{offenseB@Of}{\def\unit@offenseB@Of{#1}}
\define@key{unit}{offenseB@St}{\def\unit@offenseB@St{#1}}
\define@key{unit}{offenseB@AP}{\def\unit@offenseB@AP{#1}}
\define@key{unit}{offenserulesB}{\def\unit@offenserulesB{#1}}
\define@key{unit}{offenseweaponsB}{\def\unit@offenseweaponsB{#1}}

\define@key{unit}{offensenameC}{\def\unit@offensenameC{#1}}
\define@key{unit}{offenseC@Ag}{\def\unit@offenseC@Ag{#1}}
\define@key{unit}{offenseC@At}{\def\unit@offenseC@At{#1}}
\define@key{unit}{offenseC@Of}{\def\unit@offenseC@Of{#1}}
\define@key{unit}{offenseC@St}{\def\unit@offenseC@St{#1}}
\define@key{unit}{offenseC@AP}{\def\unit@offenseC@AP{#1}}
\define@key{unit}{offenserulesC}{\def\unit@offenserulesC{#1}}
\define@key{unit}{offenseweaponsC}{\def\unit@offenseweaponsC{#1}}
\define@key{unit}{offenseCcomment}{\def\unit@offenseCcomment{#1}}

\define@key{unit}{offensenameI}{\def\unit@offensenameI{#1}}
\define@key{unit}{offenseI@Ag}{\def\unit@offenseI@Ag{#1}}
\define@key{unit}{offenseI@St}{\def\unit@offenseI@St{#1}}
\define@key{unit}{offenseI@AP}{\def\unit@offenseI@AP{#1}}
\define@key{unit}{offenserulesI}{\def\unit@offenserulesI{#1}}
\define@key{unit}{offenseweaponsI}{\def\unit@offenseweaponsI{#1}}

\define@key{unit}{magic}{\def\unit@magic{#1}}
\define@key{unit}{magicnoptstag}{\def\unit@magicnoptstag{#1}}
\define@key{unit}{paths}{\def\unit@paths{#1}}
\define@key{unit}{pathsnote}{\def\unit@pathsnote{#1}}
\define@key{unit}{wizardconclave}{\def\unit@wizardconclave{#1}}
\define@key{unit}{options}{\def\unit@options{#1}}
\define@key{unit}{explicitoptions}{\def\unit@explicitoptions{#1}}
\define@key{unit}{mounts}{\def\unit@mounts{#1}}
\define@key{unit}{commandgroup}{\def\unit@commandgroup{#1}}
\define@key{unit}{additionalframe}{\def\unit@additionalframe{#1}}
\define@key{unit}{toggles}{\def\unit@toggles{#1}}


\define@key{unit}{modelrulesdef}{\ifdefined\thisisascrollcompendium\else\def\unit@modelrulesdef{#1}\fi}
\define@key{unit}{optionalmodelrulesdef}{\ifdefined\thisisascrollcompendium\else\def\unit@optionalmodelrulesdef{#1}\fi}
\define@key{unit}{endtext}{\ifdefined\thisisascrollcompendium\else\def\unit@endtext{#1}\fi}
\define@key{unit}{scrollmodelrulesdef}{\ifdefined\thisisascrollcompendium\def\unit@modelrulesdef{#1}\fi}
\define@key{unit}{scrolloptionalmodelrulesdef}{\ifdefined\thisisascrollcompendium\def\unit@optionalmodelrulesdef{#1}\fi}
\define@key{unit}{scrollendtext}{\ifdefined\thisisascrollcompendium\def\unit@endtext{#1}\fi}


\define@key{unit}{flavour}{\ifdefined\thisisascrollcompendium\def\unit@flavour{#1}\else\def\unit@flavour{}\fi}

\define@key{unit}{nostats}{\def\unit@nostats{#1}}
\define@key{unit}{noheader}{\def\unit@noheader{#1}}

\define@key{unit}{loadparentunit}{\def\unit@loadparentunit{#1}}

%\define@key{unit}{scrollmodelrulesdef}{\def\unit@scrollmodelrulesdef{#1}}
%\define@key{unit}{scrolloptionalmodelrulesdef}{\def\unit@scrolloptionalmodelrulesdef{#1}}


%%% Characteristics table definition %%%

\newcommand{\startcharacteristicstable}{%
\begin{tabular}{@{}p{0.21\textwidth}@{}P{0.06\textwidth}@{}P{0.06\textwidth}@{}P{0.06\textwidth}@{}P{0.06\textwidth}@{}P{0.06\textwidth}@{}p{0.49\textwidth}@{}}%
\rowcolor{\lightgreycolor}[][]%
}

\newcommand{\characteristicstableglobalheader}[1]{
	% #1 is unit's Rsr value
	\hspace*{3pt}{\ChLab{\GlobalCharacteristics}}&%
	{\ChLab{\AdvanceRateInitials}}&%
	{\ChLab{\MarchRateInitials}}&%
	{\ChLab{\DisciplineInitials}}&%
	\expandafter\ifblank\expandafter{#1}{%
		&&%
	}{%
		\iftoggle{twocellsRsr}{%
			\multicolumn{2}{@{}P{0.12\textwidth}@{}}{\ChLab{\resurrectedInitials}}&%
		}{%
			\ChLab{\resurrectedInitials}&&%
		}%
	}%
	\ChLab{\ModelRules}%
	\tabularnewline%
}

\newcommand{\characteristicstableglobalrow}[8]{%
% 1     2   3   4   5            6      7      8
% name, Ad, Ma, Di, globalrules, Adfly, Mafly, Rsr
	\expandafter\ifblank\expandafter{#1}{}{\nameindent#1}%
	\expandafter\ifblank\expandafter{#6}{}{%
		\strut\hspace*{\fill}\setlength{\parskip}{0pt}%
		\ChLab{\groundmovementlabel}\par%
		\strut\hspace*{\fill}\ChLab{\flymovementlabel}%
	}%
	&%
	\ChValDistance{#2}
	\expandafter\ifblank\expandafter{#6}{}{\setlength{\parskip}{0pt}\par\ChVal{\distance{#6}}}%
	&%
	\ChValDistance{#3}
	\expandafter\ifblank\expandafter{#7}{}{\setlength{\parskip}{0pt}\par\ChVal{\distance{#7}}}%
	&%
	\ChVal{#4}&%
	\expandafter\ifblank\expandafter{#8}{&&}{%
		\iftoggle{twocellsRsr}{%
			\multicolumn{2}{@{}P{0.12\textwidth}@{}}{\ChVal{#8}}&%
		}{%
			\ChVal{#8}&&%
		}%
	}%
	\strut\expandafter\ifblank\expandafter{#5}{}{\expandafter\alphaorderlist\expandafter{#5}}%
	\tabularnewline%
}

\newcommand{\characteristicstabledefensiveheader}[2]{
	% #1 is unit's Aeg value
	% #2 is unit's printAeg value
	\hspace*{3pt}\ChLab{\DefensiveCharacteristics{}}&%
	\ChLab{\HealthPointsInitials}&%
	\ChLab{\DefensiveSkillInitials}&%
	\ChLab{\ResilienceInitials}& %
	\ChLab{\ArmourInitials}&%
	\expandafter\ifblank\expandafter{#2}{%
		\expandafter\ifblank\expandafter{#1}{}{\ChLab{\AegisInitials}}%
    }{\ChLab{\AegisInitials}}&%
    \tabularnewline%
}

\newcommand{\characteristicstabledefensiverow}[8]{%
% 1     2   3   4   5    6    7             8
% name, HP, Df, Re, Arm, Aeg, defenserules, defensearmour
	\expandafter\ifblank\expandafter{#1}{}{\nameindent#1}&%
	\ChVal{#2}&%
	\ChVal{#3}&%
	\ChVal{#4}&%
	\ChValZero{#5}&%
	\ChVal{#6}&%
	\strut\expandafter\ifblank\expandafter{#7}{}{\expandafter\alphaorderlist\expandafter{#7}}%
	\expandafter\ifblank\expandafter{#8}{}{%
		\expandafter\ifblank\expandafter{#7}{}{, }%
		\expandafter\alphaorderlist\expandafter{#8}%
	}%
	\tabularnewline%
}

\newcommand{\characteristicstableoffensiveheader}{
	\hspace*{3pt}\ChLab{\OffensiveCharacteristics{}}&%
	\ChLab{\AttackValueInitials}&%
	\ChLab{\OffensiveSkillInitials}&%
	\ChLab{\StrengthInitials}& %
	\ChLab{\ArmourPenetrationInitials}&%
	\ChLab{\AgilityInitials}&%
	\tabularnewline%	
}

\newcommand{\characteristicstableoffensiverow}[9]{%
% 1     2   3   4   5   6   7             8               9
% name, At, Of, St, AP, Ag, offenserules, offenseweapons, offensecomment
	\nameindent#1&%  now we walways print offense names
	\ChVal{#2}&%
	\ChVal{#3}&%
	\ChVal{#4}&%	
	\ChVal{#5}&%
	\ChVal{#6}&%
	\strut\expandafter\ifblank\expandafter{#7}{}{\expandafter\alphaorderlist\expandafter{#7}}%
	\expandafter\ifblank\expandafter{#8}{}{%
		\expandafter\ifblank\expandafter{#7}{}{, }%
		\expandafter\alphaorderlist\expandafter{#8}%
	}%
	\expandafter\ifblank\expandafter{#9}{}{\newline#9}%
	\tabularnewline%
}



%%% Length definition for the category note

\newlength{\categorynoteheight}%


%%% Length definition for the name block when there is the English name to add

\newlength{\maxnamelength}%
\newlength{\currentnamelength}%


%%% Toggle for two cells for Rsr

\newtoggle{twocellsRsr}
\togglefalse{twocellsRsr}


%%% Fluff box definition %%%

\newtoggle{insideunitentry}
\togglefalse{insideunitentry}

\newcommand{\fluffbox}[1]{%
	\iftoggle{insideunitentry}{%
		\vspace*{-2pt}\hspace*{3pt}%
		\begin{minipage}{\textwidth-6pt}%
			\itshape%
			#1%
		\end{minipage}%
		\vspace*{6pt}\par%
	}{%
		\def\@minipagerestore{\setlength{\parskip}{\mycurrentparskip}}% Else paragraph and title spaces behave differently when inside or outside a minipage - Unit entry spacing has been coded with this in mind, if you remove it, all spaces will be off
		\vspace*{5pt}\hspace*{5pt}%
		\fboxsep5pt
		\colorbox{\lightgreycolor}{%
			\begin{minipage}{\textwidth-20pt}%
				\itshape%
				#1%
			\end{minipage}%
		}%
		\vspace*{5pt}\par%
		\ifdefined\thisisanarmybook%
			%
		\else%
			\def\@minipagerestore{}%
		\fi%
	}%
}

\newcommand{\fluffref}[1]{\vspace*{3pt}\newline\hspace*{0.5\textwidth}\begin{minipage}{0.5\textwidth}\begin{flushright}\textnormal{---#1}\end{flushright}\end{minipage}}
\newcommand{\fluffrefnonewline}[1]{\vspace*{3pt}\hspace*{0.5\textwidth}\begin{minipage}{0.5\textwidth}\begin{flushright}\textnormal{---#1}\end{flushright}\end{minipage}}

%%% Command to add a new unit definition %%%

\newcommand{\defunit}{
	\setkeys{unit}{%
		subentrynumber=, name=, QRSname=, logo=, speciallogo=, secondlogo=, optionallogo=, secondoptionallogo=, thirdoptionallogo=, cost=, unitsize=, costpermodel=, maxunitsize=, maxunitsperarmy=, maxmodelsperarmy=, maxmountsperarmy=, addrestriction=, type=, size=, basesize=, basesizenotag=, scoring=, categorynote=, costnote=, fluff=, globalname=, global@Ad=, global@Adfly=, global@Ma=, global@Mafly=, global@Di=, global@Rsr=, globalrules=, globalnameB=, globalB@Ad=, globalB@Adfly=, globalB@Ma=, globalB@Mafly=, globalB@Di=, globalB@Rsr=, globalrulesB=, globalnameC=, globalC@Ad=, globalC@Adfly=, globalC@Ma=, globalC@Mafly=, globalC@Di=, globalC@Rsr=, globalrulesC=, defensename=, defense@HP=, defense@Df=, defense@Re=, defense@Arm=, defense@Aeg=, printAeg=, defenserules=, defensearmour=, defensenameB=, defenseB@HP=, defenseB@Df=, defenseB@Re=, defenseB@Arm=, defenseB@Aeg=, printAegB=, defenserulesB=, defensearmourB=, defensenameC=, defenseC@HP=, defenseC@Df=, defenseC@Re=, defenseC@Arm=, defenseC@Aeg=, printAegC=, defenserulesC=, defensearmourC=, offensename=, forceoffensenameprint=, offense@Ag=, offense@At=, offense@Of=,  offense@St=, offense@AP=, offenserules=, offenseweapons=, offensecomment=, offensenameB=, offenseB@Ag=, offenseB@At=, offenseB@Of=,  offenseB@St=, offenseB@AP=, offenserulesB=, offenseweaponsB=, offensenameC=, offenseC@Ag=, offenseC@At=, offenseC@Of=,  offenseC@St=, offenseC@AP=, offenserulesC=, offenseweaponsC=, offenseCcomment=, offensenameI=, offenseI@Ag=, offenseI@St=, offenseI@AP=, offenserulesI=, offenseweaponsI=, magic=, magicnoptstag=, paths=, pathsnote=, wizardconclave=, options=, explicitoptions=, mounts=, commandgroup=, modelrulesdef=, optionalmodelrulesdef=, additionalframe=, endtext=, toggles=, scrollmodelrulesdef=, scrolloptionalmodelrulesdef=, scrollendtext=, flavour=, nostats=, noheader=, loadparentunit=,
	}%
	\setkeys{unit}%
}

\newcommand{\additionalframename}{NO NAME DEFINED}

%%% Counters and boolean for two columns balance %%%

\newtoggle{dontdoanythingjustload} % used for loading data without displaying

\newtoggle{printoffensename}

\newcounter{numberofnonrulesdefframes}
\newcounter{numberofrestrictions}
\newcounter{numberoftwocolframes}

\newtoggle{thereismagic}
\newtoggle{thereiswizardconclave}
\newtoggle{thereisoptions}
\newtoggle{thereismounts}
\newtoggle{thereiscommandgroup}
\newtoggle{thereismodelrulesdef}
\newtoggle{thereisoptionalmodelrulesdef}
\newtoggle{thereisadditionalframe}

\newtoggle{twocolprintoptions}
\newtoggle{twocolprintmounts}
\newtoggle{twocolprintcommandgroup}
\newtoggle{twocolprintmodelrulesdef}
\newtoggle{twocolprintoptionalmodelrulesdef}
\newtoggle{twocolprintadditionalframe}

\newtoggle{onecolprintoptions}
\newtoggle{onecolprintmounts}
\newtoggle{onecolprintcommandgroup}
\newtoggle{onecolprintmodelrulesdef}
\newtoggle{onecolprintoptionalmodelrulesdef}
\newtoggle{onecolprintadditionalframe}

\newtoggle{thereismorethanXitemsinoptions}
\newtoggle{optionsistheonlynonrulesdefframe}

\newtoggle{show_twocolframes}
\newtoggle{show_oncolframes}
\newtoggle{thereisatleastoneframe}


%%% The Command %%%

\newlength{\startatthetopoftheline}
\setlength{\startatthetopoftheline}{0.5pt-\baselineskip}
\newlength{\headerlogosize}%
\newlength{\headerlogoslength}%

\newcommand{\unitentryheader}{%
\vspace*{\startatthetopoftheline}\rule{\textwidth}{0.5pt}\par\vspace{-5pt}%
% --------------------------------------------
{%
\setlength{\headerlogosize}{0.08\textwidth}%
% setting up the layout based on the number of logos
\expandafter\ifblank\expandafter{\unit@secondlogo}{% 
	% one logo
	\setlength{\headerlogoslength}{0.09\textwidth}%
}{%
	% two logos
	\setlength{\headerlogoslength}{0.17\textwidth}%
	\renewcommand{\perextramodel}{\perextramodelshortened}%
	\setlength{\lengthbeforemodelsize}{\lengthbeforemodelsize - 0.08\textwidth}%
}%
% Logos
\hspace*{-3pt}%
\begin{minipage}[b]{\headerlogoslength}%
\expandafter\ifblank\expandafter{\unit@logo}{% is there a normal logo?
	\expandafter\ifblank\expandafter{\unit@speciallogo}{%
		\rule{0pt}{1.2cm}% ensuring the banner is as high as when we have a picture, and that the second minipage is centered approprietly
	}{%
		\begin{center}\includegraphics[width=0.7\headerlogosize]{\unit@speciallogo}\end{center}%
	}%
}{%
	\expandafter\ifblank\expandafter{\unit@logo}{}{%
		\includegraphics[width=\headerlogosize]{{logo_\unit@logo}.png}}%	
	\expandafter\ifblank\expandafter{\unit@secondlogo}{}{%
		\includegraphics[width=\headerlogosize]{{logo_\unit@secondlogo}.png}}%
}%
\vspace*{-4pt}%	
\end{minipage}%
% Title, unit cost
\setlength{\maxnamelength}{0.59\textwidth-\headerlogoslength+3pt}
\begin{minipage}[b]{\maxnamelength}%
%
% Title with hyperlink
{\Largerfontsize\textbf{%
	\expandafter\ifblank\expandafter{\unit@subentrynumber}{}{(\unit@subentrynumber{}) }%
	% Should we add an English name after the standard name of the unit?
	\def\tempENname{}%
	\ConvertNameMacroToStr{\unit@name}{\tempENname}%
	\CombineStringsIntoMacroAndExpand{\tempENname}{EN}{\tempENname}%
	\expandafter\ifblank\expandafter{\tempENname}{% Then there is no English name to add
		\unit@name%
	}{% There is an English name to add
		% Now see if the fontsize need to be decreased for everything to fit on a line
		\settowidth{\currentnamelength}{\Largefontsize\unit@name{} \textnormal{\largerfontsize\textit{(\tempENname)}}}%
		\ifdim\currentnamelength>\maxnamelength\relax% Name's too big we need to reduce the fontsize
			{\largerfontsize\unit@name{} \textnormal{\largefontsize\textit{(\tempENname)}}}%
		\else% Everything is good
			{\Largefontsize\unit@name{} \textnormal{\largerfontsize\textit{(\tempENname)}}}%
		\fi%
	}%
}}%
%
% Cost and the number of models
\vspace*{3pt}\newline%
% Is everything replace by a note?
\expandafter\ifblank\expandafter{\unit@costnote}{% NO
	% Unit cost and size
	\expandafter\ifblank\expandafter{\unit@cost}{}{{\pointsblock{\largerfontsize\pts{\textbf{\unit@cost}}}} }%
	% Is it a single model or a unit with multiple models?
	\expandafter\ifblank\expandafter{\unit@unitsize}{}{% if there is no unitsize, we don't print anything at all
		% Now we want to see if there is 1 or more models
		\isitoneornot{\unit@unitsize}{% 1 model
			% Can you add models?
			\expandafter\ifblank\expandafter{\unit@costpermodel}{% no other models
				\hspace*{\lengthbeforemodelsize}%
				\setlength{\lengthoftheptsblock}{\widthof{{\largerfontsize\pts{\textbf{\unit@cost}}} }}%
				\hspace*{-\lengthoftheptsblock}%
				\ifdef{\languageisrussian}{\hspace*{-0.5cm}}{}%
				\singlemodel{}%
			}{% possibility of more models
				\pointsblock{+ \pts{\textbf{\unit@costpermodel}}\perextramodel{}}%
				%
				\hspace*{\lengthbeforemodelsize}%
				\setlength{\lengthoftheptsblock}{\widthof{{\largerfontsize\pts{\textbf{\unit@cost}}} + \pts{\textbf{\unit@costpermodel}}\perextramodel{}}}%
				\hspace*{-\lengthoftheptsblock}%
				\textbf{1--\unit@maxunitsize} \Models{}%
			}%
		}{% more than 1 model
			\pointsblock{+ \pts{\textbf{\unit@costpermodel}}\perextramodel{}}%
			%
			\hspace*{\lengthbeforemodelsize}%
			\setlength{\lengthoftheptsblock}{\widthof{{\largerfontsize\pts{\textbf{\unit@cost}}} + \pts{\textbf{\unit@costpermodel}}\perextramodel{}}}%
			\hspace*{-\lengthoftheptsblock}%
			\textbf{\unit@unitsize{}--\unit@maxunitsize} \Models{}%
		}%
	}%
}{% everything replaced with a cost note
	\unit@costnote{}%
}%
\vspace{0pt}%
\end{minipage}%
%
% Restrictions and Scoring Logo
\hfill\begin{minipage}[b]{0.2\textwidth}%
\setlength{\parskip}{0pt}%
\begin{center}%
% Count the number of restrictions
\setcounter{numberofrestrictions}{0}% reset
\expandafter\ifblank\expandafter{\unit@maxunitsperarmy}{}{\stepcounter{numberofrestrictions}}%
\expandafter\ifblank\expandafter{\unit@maxmountsperarmy}{}{\stepcounter{numberofrestrictions}}%
\expandafter\ifblank\expandafter{\unit@maxmodelsperarmy}{}{\stepcounter{numberofrestrictions}}%
\expandafter\ifblank\expandafter{\unit@addrestriction}{}{\stepcounter{numberofrestrictions}}%
\expandafter\ifstrequal\expandafter{\unit@scoring}{yes}{%
	\ifnumcomp{\value{numberofrestrictions}}{>}{1}{%
		\strut\includegraphics[height=8pt]{logo_scoring.png}\par%
	}{%
		\ifnumcomp{\value{numberofrestrictions}}{=}{0}{%
			\strut\vspace*{6pt}\includegraphics[height=12pt]{logo_scoring.png}\par%
		}{%
			\strut\includegraphics[height=12pt]{logo_scoring.png}\par%
		}%
	}%
}{}%
\expandafter\ifblank\expandafter{\unit@maxunitsperarmy}{}{%
	\zerotoXunitsperarmy{\unit@maxunitsperarmy}\par%
}%
\expandafter\ifblank\expandafter{\unit@maxmountsperarmy}{}{%
	\zerotoXmountsperarmy{\unit@maxmountsperarmy}\par%
}%
\expandafter\ifblank\expandafter{\unit@maxmodelsperarmy}{}{%
	\zerotoXmodelsperarmy{\unit@maxmodelsperarmy}\par%
}%
\expandafter\ifblank\expandafter{\unit@addrestriction}{}{%
	\unit@addrestriction%
}%
\end{center}%
\vspace{0pt}%
\end{minipage}%
%
% Size, Type and Base
\begin{minipage}[b]{0.2\textwidth}%
\renewcommand{\arraystretch}{1}%
\begin{tabular}{@{}>{\raggedleft}p{0.25\textwidth}@{\hskip 0.03\textwidth}p{0.72\textwidth}@{}}%
\ChLab{\Height}&\unit@size{}\tabularnewline%
\ChLab{\type}&\unit@type{}\tabularnewline%
\ChLab{\basesize}&\unit@basesize{}%
\expandafter\ifblank\expandafter{\unit@basesizenotag}{%
	~\mm%
	\ifsubstring{\unit@basesize}{\timess}{}{~\roundbase}%
}{}%
\tabularnewline%
\end{tabular}%
\vspace*{-4pt}%
\end{minipage}\par%
}% end of title section
\vspace{-10pt}\rule{\textwidth}{0.5pt}\par% 
% --------------------------------------------
%
% Category note and optional logo
\expandafter\ifblank\expandafter{\unit@categorynote}{%
% if no Cetegory note, skip this section
}{% 
\expandafter\ifblank\expandafter{\unit@optionallogo}{% is there a logo?
	% no logos
    \vspace*{-4pt}%
    \unit@categorynote{}%
    \par\vspace*{-10pt}%	
}{%
	% set the space based on the number of logos
	\setlength{\headerlogosize}{0.06\textwidth}%
	\expandafter\ifblank\expandafter{\unit@secondoptionallogo}{% 
		% one logo
		\setlength{\headerlogoslength}{0.06\textwidth}%
	}{%
		\expandafter\ifblank\expandafter{\unit@thirdoptionallogo}{% 
			% two logos
			\setlength{\headerlogoslength}{0.12\textwidth+2pt}%
		}{%
			% three logos
			\setlength{\headerlogoslength}{0.18\textwidth+3pt}%
	}}%
	% Check if the category note has more than 2 lines.
	\setbox0=\vbox{\begin{minipage}{\textwidth-\headerlogoslength}\unit@categorynote{}\end{minipage}}%
	\categorynoteheight=\ht0 \advance\categorynoteheight by \dp0\relax% Two lines do approx. a bit less than 20 pt
	%
	% Optional Logos
	%
	\ifdim\categorynoteheight>25pt\relax
		\vspace*{-1pt}%
	\else
		\vspace*{-7pt}%
	\fi
	\begin{minipage}{\headerlogoslength}%
		\expandafter\ifblank\expandafter{\unit@optionallogo}{}{
			\includegraphics[width=\headerlogosize]{{logo_\unit@optionallogo}.png}}%
		\expandafter\ifblank\expandafter{\unit@secondoptionallogo}{}{
			\includegraphics[width=\headerlogosize]{{logo_\unit@secondoptionallogo}.png}}%
		\expandafter\ifblank\expandafter{\unit@thirdoptionallogo}{}{
			\includegraphics[width=\headerlogosize]{{logo_\unit@thirdoptionallogo}.png}}%
	\end{minipage}%
	%
	% Category Note
	%
	\begin{minipage}{\textwidth-\headerlogoslength}%
		\unit@categorynote{}%
	\end{minipage}%
	\par%
	\ifdim\categorynoteheight>25pt\relax
		\vspace*{-2pt}%
	\else
	    \vspace*{-6pt}%
	\fi
	}% End cetegory note with logos sections
%
\rule{\textwidth}{0.5pt}\par% 
% --------------------------------------------
}% End cetegory note sections
%
% Fluff
\ifdefvoid{\unit@fluff}{}{%
	\ifundef{\nofluffinunitentry}{%
		\fluffbox{\unit@fluff}%
		\vspace{-10pt}\rule{\textwidth}{0.5pt}\par%
		% --------------------------------------------
	}{}%
}%
}% END \unitentryheader


\newcommand{\unitentry}[1]{%
\toggletrue{insideunitentry}%
\def\@minipagerestore{\setlength{\parskip}{\mycurrentparskip}}% Else paragraph and title spaces behave differently when inside or outside a minipage - Unit entry spacing has been coded with this in mind, if you remove it, all spaces will be off
\defunit{#1}%
%
\iftoggle{dontdoanythingjustload}{}{
%
\begin{minipage}[t]{\textwidth}% To avoid splitted profiles between 2 pages
%
% load the parent unitentry
\expandafter\ifblank\expandafter{\unit@loadparentunit}{}{%
	\toggletrue{dontdoanythingjustload}%
	\unit@loadparentunit% load the parent
	\setkeys{unit}{#1}% need to load this unit again
	\togglefalse{dontdoanythingjustload}% 
	}%   
% default values
\expandafter\ifsubstring\expandafter{\unit@globalrules}{\scoring}{%
    \setkeys{unit}{scoring=yes}}{}%
\expandafter\ifblank\expandafter{\unit@offensename}{%
	\setkeys{unit}{offensename=\unit@name}}{}%    
%%%%%%%%%%%%%%%%%%%%%%
% Unit title, logos and general info
%
\if T\unit@noheader{}\else{%
\def\temphypertag{}%
\ConvertNameMacroToStr{\unit@name}{\temphypertag}% remove first the \ to the left and then the {} to the right
\xdef\unithypertag{\bookprefix\arabic{categorynumber}\temphypertag}% using the unit name macro to define an hypertag
\hypertarget{\unithypertag}{}%
\unitentryheader{}%
}\fi% \unit@noheader
%%%%%%%%%%%%%%%%%%%%%%
% Characteristics and rules
%%%%%%%%%%%%%%%%%%%%%%
\renewcommand{\arraystretch}{1.4}%
\if T\unit@nostats{}\else{%
% Global Characteristics
\vspace*{-7pt}%
\startcharacteristicstable{}%
\characteristicstableglobalheader{\unit@global@Rsr}%
%
\characteristicstableglobalrow{\unit@globalname}{\unit@global@Ad}{\unit@global@Ma}{\unit@global@Di}{\unit@globalrules}{\unit@global@Adfly}{\unit@global@Mafly}{\unit@global@Rsr}%
%
\expandafter\ifblank\expandafter{\unit@globalnameB}{}{%
	\characteristicstableglobalrow{\unit@globalnameB}{\unit@globalB@Ad}{\unit@globalB@Ma}{\unit@globalB@Di}{\unit@globalrulesB}{\unit@global@AdflyB}{\unit@global@MaflyB}{\unit@global@RsrB}}%
%
\expandafter\ifblank\expandafter{\unit@globalnameC}{}{%
	\characteristicstableglobalrow{\unit@globalnameC}{\unit@globalC@Ad}{\unit@globalC@Ma}{\unit@globalC@Di}{\unit@globalrulesC}{\unit@global@AdflyC}{\unit@global@MaflyC}{\unit@global@RsrC}}%
\end{tabular}%

%%%%%%%%%%%%%%%%%%%%%%
% Defensive Characteristics
\vspace*{-7pt}%
\startcharacteristicstable{}%
\characteristicstabledefensiveheader{\unit@printAeg}{\unit@defense@Aeg}
\characteristicstabledefensiverow{\unit@defensename}{\unit@defense@HP}{\unit@defense@Df}{\unit@defense@Re}{\unit@defense@Arm}{\unit@defense@Aeg}{\unit@defenserules}{\unit@defensearmour}
%
\expandafter\ifblank\expandafter{\unit@defensenameB}{}{%
	\characteristicstabledefensiverow{\unit@defensenameB}{\unit@defenseB@HP}{\unit@defenseB@Df}{\unit@defenseB@Re}{\unit@defenseB@Arm}{\unit@defenseB@Aeg}{\unit@defenserulesB}{\unit@defensearmourB}}%
%
\expandafter\ifblank\expandafter{\unit@defensenameC}{}{%
	\characteristicstabledefensiverow{\unit@defensenameC}{\unit@defenseC@HP}{\unit@defenseC@Df}{\unit@defenseC@Re}{\unit@defenseC@Arm}{\unit@defenseC@Aeg}{\unit@defenserulesC}{\unit@defensearmourC}}%
\end{tabular}%

%%%%%%%%%%%%%%%%%%%%%%
% Offensive Characteristics
\vspace*{-7pt}%
\startcharacteristicstable{}%
\characteristicstableoffensiveheader{}
\characteristicstableoffensiverow{\unit@offensename}{\unit@offense@At}{\unit@offense@Of}{\unit@offense@St}{\unit@offense@AP}{\unit@offense@Ag}{\unit@offenserules}{\unit@offenseweapons}{\unit@offensecomment}
%
\expandafter\ifblank\expandafter{\unit@offensenameB}{}{%
	\characteristicstableoffensiverow{\unit@offensenameB}{\unit@offenseB@At}{\unit@offenseB@Of}{\unit@offenseB@St}{\unit@offenseB@AP}{\unit@offenseB@Ag}{\unit@offenserulesB}{\unit@offenseweaponsB}{}}% No such thing as offenseBcomment
%
\expandafter\ifblank\expandafter{\unit@offensenameC}{}{%
	\characteristicstableoffensiverow{\unit@offensenameC}{\unit@offenseC@At}{\unit@offenseC@Of}{\unit@offenseC@St}{\unit@offenseC@AP}{\unit@offenseC@Ag}{\unit@offenserulesC}{\unit@offenseweaponsC}{\unit@offenseCcomment}}%
%
\expandafter\ifblank\expandafter{\unit@offensenameI}{}{%
	\characteristicstableoffensiverow{\unit@offensenameI}{}{}{\unit@offenseI@St}{\unit@offenseI@AP}{\unit@offenseI@Ag}{\unit@offenserulesI}{\unit@offenseweaponsI}{}}% No such thing as I@At, I@Of, offenseIcomment
\end{tabular}\newline%
\vspace*{-8pt}\strut% Fixing the space after the characteristics table (else there is a bug when there is multiple offense lines)

}\fi% end stats table \unit@nostats{}
%
%%%%%%%%%%%%%%%%%%%%%%
% Assessing the balance of the columns to see if we should put some frames outside of the two column layout
%%%%%%%%%%%%%%%%%%%%%%
% Which frames are defined?

\expandafter\ifblank\expandafter{\unit@options}{}{\toggletrue{thereisoptions}}%
\expandafter\ifblank\expandafter{\unit@explicitoptions}{}{\toggletrue{thereisoptions}}%
\expandafter\ifblank\expandafter{\unit@magic}{}{\toggletrue{thereismagic}}%
\expandafter\ifblank\expandafter{\unit@paths}{}{\toggletrue{thereismagic}}%
\expandafter\ifblank\expandafter{\unit@wizardconclave}{}{\toggletrue{thereiswizardconclave}}%
\expandafter\ifblank\expandafter{\unit@mounts}{}{\toggletrue{thereismounts}}%
\expandafter\ifblank\expandafter{\unit@commandgroup}{}{\toggletrue{thereiscommandgroup}}%
\expandafter\ifblank\expandafter{\unit@modelrulesdef}{}{\toggletrue{thereismodelrulesdef}}%
\expandafter\ifblank\expandafter{\unit@optionalmodelrulesdef}{}{\toggletrue{thereisoptionalmodelrulesdef}}%
\expandafter\ifblank\expandafter{\unit@additionalframe}{}{\toggletrue{thereisadditionalframe}}%



% Are there any non rules-def frames? If there is only one, is it options?
% We need to decide whatever to create a two-column environment
\setcounter{numberofnonrulesdefframes}{0}% reset
\iftoggle{thereismagic}{\stepcounter{numberofnonrulesdefframes}}{}%
\iftoggle{thereiswizardconclave}{\stepcounter{numberofnonrulesdefframes}}{}%
\iftoggle{thereismounts}{\stepcounter{numberofnonrulesdefframes}}{}%
\iftoggle{thereiscommandgroup}{\stepcounter{numberofnonrulesdefframes}}{}%
\iftoggle{thereisadditionalframe}{\stepcounter{numberofnonrulesdefframes}}{}%
\iftoggle{thereisoptions}{\stepcounter{numberofnonrulesdefframes}%
	\ifnumcomp{\value{numberofnonrulesdefframes}}{=}{1}{% then there is only options defined
		\toggletrue{optionsistheonlynonrulesdefframe}%
	}{}% End of \ifnumcomp
}{}% End of \iftoggle{thereisoptions}
%
% Is options long enough to put it out of the twocols layout?
\iftoggle{thereisoptions}{%
	\expandafter\ifblank\expandafter{\unit@options}{\toggletrue{twocolprintoptions}}{% For specific options, we don't have a way yet to count the options, so we set the two cols layout by default
	\expandafter\countitemsinoptionslist\expandafter{\unit@options}%
	\ifnumcomp{\value{numberofitemsinlist}}{>}{3}{\toggletrue{thereismorethanXitemsinoptions}}{}%
	\ifboolexpr{ togl {optionsistheonlynonrulesdefframe} and togl {thereismorethanXitemsinoptions}}{%
		\toggletrue{onecolprintoptions}
	}{\toggletrue{twocolprintoptions}}% End ifnumcomp
	}%
}{}% End of \iftoggle{thereisoptions}
%
% How do we handle the modelrulesdef now?
\ifboolexpr{ test {\ifnumcomp{\value{numberofnonrulesdefframes}}{=}{0}}
	or
	(test {\ifnumcomp{\value{numberofnonrulesdefframes}}{=}{1}} and togl {onecolprintoptions})
}{% Then we want rules on a one col layout
	\iftoggle{thereismodelrulesdef}{\toggletrue{onecolprintmodelrulesdef}}{}%
	\iftoggle{thereisoptionalmodelrulesdef}{\toggletrue{onecolprintoptionalmodelrulesdef}}{}%
}{% Else we want them in twocols layout
	\iftoggle{thereismodelrulesdef}{\toggletrue{twocolprintmodelrulesdef}}{}%
	\iftoggle{thereisoptionalmodelrulesdef}{\toggletrue{twocolprintoptionalmodelrulesdef}}{}%
}% End of \ifboolexpr
%
% How do we handle the last frames?
\setcounter{numberoftwocolframes}{0}%
\iftoggle{thereismagic}{\stepcounter{numberoftwocolframes}}{}%
\iftoggle{thereiswizardconclave}{\stepcounter{numberoftwocolframes}}{}%
\iftoggle{thereismounts}{\stepcounter{numberoftwocolframes}\toggletrue{twocolprintmounts}}{}%
\iftoggle{thereisadditionalframe}{\stepcounter{numberoftwocolframes}\toggletrue{twocolprintadditionalframe}}{}%
\iftoggle{twocolprintoptions}{\stepcounter{numberoftwocolframes}}{}%
\iftoggle{thereiscommandgroup}{%
	\ifnumcomp{\value{numberoftwocolframes}}{>}{0}{%
		\toggletrue{twocolprintcommandgroup}%
	}{%
		\expandafter\countitemsinoptionslist\expandafter{\unit@commandgroup}%
		\ifnumcomp{\value{numberofitemsinlist}}{>}{1}{%
				\toggletrue{onecolprintcommandgroup}%
		}{%
				\toggletrue{twocolprintcommandgroup}%
		}%
	}%
}{}%

	
%%%%%%%%%%%%%%%%%%%%%%
% If any toggles have to be manually changed
\unit@toggles{}%
\ifdefined\thisisascrollcompendium
	\expandafter\ifblank\expandafter{\unit@modelrulesdef}{}{
		\toggletrue{onecolprintmodelrulesdef}%
		\togglefalse{twocolprintmodelrulesdef}%
	}
		\expandafter\ifblank\expandafter{\unit@optionalmodelrulesdef}{}{
		\toggletrue{onecolprintoptionalmodelrulesdef}%
		\togglefalse{twocolprintoptionalmodelrulesdef}%
	}
\fi
% frames that always occupy half page
%    magic
%    wizardconclave
%
% frames that can take both half or a full page
%    printadditionalframe
%    options 
%    mounts
%    commandgroup
%    modelrulesdef
%    optionalmodelrulesdef
%
\ifboolexpr{ 
       togl{twocolprintmodelrulesdef}
	or togl{twocolprintoptions}
	or togl{thereismagic}
	or togl{thereiswizardconclave}
	or togl{twocolprintmounts}
	or togl{twocolprintcommandgroup}
	or togl{twocolprintoptionalmodelrulesdef}
	or togl{twocolprintadditionalframe}
	}{\toggletrue{show_twocolframes}
	}{}
\ifboolexpr{  
       togl {onecolprintmodelrulesdef}
	or togl {onecolprintoptions}
	or togl {onecolprintmounts}
	or togl {onecolprintcommandgroup}
	or togl {onecolprintadditionalframe}
	or togl {onecolprintoptionalmodelrulesdef}
	}{\toggletrue{show_oncolframes}
	}{}

% Is there at least one thing to put in the big rules frame?
\ifboolexpr{ 
 	   togl {show_oncolframes}
	or togl {show_twocolframes}
}{\toggletrue{thereisatleastoneframe}}{}%

%%%%%%%%%%%%%%%%%%%%%%
% Rules frame
% We print the frame if and only if there is something to put inside
\iftoggle{thereisatleastoneframe}{%
	%%%%%%%%%%%%%%%%%%%%%%
	% Two columns frames, we print it if there is something to put in the frame
	\ifboolexpr{ togl{show_twocolframes}
	}{%
		\setlength{\columnsep}{10pt}%
		\setlength{\multicolsep}{0pt}%
		\raggedcolumns%
		\begin{multicols}{2}%
			\iftoggle{twocolprintmodelrulesdef}{\strut\modelrulesdef{\unit@modelrulesdef}}{}%
			\iftoggle{thereismagic}{%
				\expandafter\ifblank\expandafter{\unit@magicnoptstag}{%
					\strut\magic{\unit@magic}{\unit@paths}{\unit@pathsnote}%
				}{%
					\strut\magicnoptstag{\unit@magic}{\unit@paths}{\unit@pathsnote}%	
				}%
			}{}%
			\iftoggle{twocolprintoptions}{%
				\expandafter\ifblank\expandafter{\unit@options}{}{\strut\options{\unit@options}}%
				\expandafter\ifblank\expandafter{\unit@explicitoptions}{}{\strut\explicitoptions{\unit@explicitoptions}}%				
			}{}%
			\iftoggle{thereiswizardconclave}{\strut\printwizardconclave{\unit@wizardconclave}}{}%
			\iftoggle{twocolprintmounts}{\strut\mounts{\unit@mounts}}{}%
			\iftoggle{twocolprintcommandgroup}{\strut\commandgroup{\unit@commandgroup}}{}%
			\iftoggle{twocolprintadditionalframe}{\strut\additionalframe{\unit@additionalframe}}{}%
			\iftoggle{twocolprintoptionalmodelrulesdef}{\strut\optionalmodelrulesdef{\unit@optionalmodelrulesdef}}{}%
		\end{multicols}%
		% Additional space if there is a single column rule after
		\ifboolexpr{ togl {show_oncolframes}%
		}{\vspace*{8pt}}{}%
	}{}%
	%%%%%%%%%%%%%%%%%%%%%%
	% Single column
	\ifboolexpr{ togl {show_oncolframes}%
	}{%
		\vspace*{-5pt}%
		\iftoggle{onecolprintmodelrulesdef}{\strut\modelrulesdef{\unit@modelrulesdef}}{}%
		\iftoggle{onecolprintoptions}{%
			\expandafter\ifblank\expandafter{\unit@options}{}{\strut\optionstwocols{\unit@options}}%
			\expandafter\ifblank\expandafter{\unit@explicitoptions}{}{\strut\explicitoptionstwocols{\unit@explicitoptions}}%			
		}{}%
		\iftoggle{onecolprintmounts}{\strut\mountstwocols{\unit@mounts}}{}%
		\iftoggle{onecolprintcommandgroup}{\strut\commandgrouptwocols{\unit@commandgroup}}{}%
		\iftoggle{onecolprintadditionalframe}{\strut\additionalframetwocols{\unit@additionalframe}}{}%
		\iftoggle{onecolprintoptionalmodelrulesdef}{\strut\optionalmodelrulesdef{\unit@optionalmodelrulesdef}}{}%	  
	}{}%
}{}% END of there is at least one frame

%%%%%%%%%%%%%%%%%%%%%%
% Additional stuff
\expandafter\ifblank\expandafter{\unit@endtext}{}{\unit@endtext}%

%\expandafter\ifblank\expandafter{\unit@flavour}{}{\center{\textit{\unit@flavour}}}%
\expandafter\ifblank\expandafter{\unit@flavour}{}{\bigskip\begin{quote}%
\small\center{\textit{\unit@flavour}}%
\end{quote}}%

\vspace*{-3pt}%
\hfill{\verysmallfontsize\textcolor{white}{d}}% Else footer goes wild for some reason with ocgcolorlinks option
\end{minipage}%
\vspace{12pt}%


%%%%%%%%%%%%%%%%%%%%%%%%%%%%%%%%%%%%%%%%%%%%%%%%%%%%
%%%%%%% Filling the profiles database for the QRS
\DTLnewrow{profiles}%
\expandafter\dtbfillunitname\expandafter{\unit@name}%

% QRS name defined? if yes, fill the database
\expandafter\ifblank\expandafter{\unit@QRSname}{%
	\def\tempQRSname{}
	\ConvertNameMacroToStr{\unit@name}{\tempQRSname}%
	\CombineStringsIntoMacroAndExpand{\tempQRSname}{QRS}{\tempQRSname}%
	\expandafter\ifblank\expandafter{\tempQRSname}{
		\dtbfillunitQRSname{}%
	}{%
		\expandafter\dtbfillunitQRSname\expandafter{\tempQRSname}%
	}%
}{%
	\expandafter\dtbfillunitQRSname\expandafter{\unit@QRSname}%
}%

\expandafter\dtbfillhypertag\expandafter{\unithypertag}%
\edef\categorynumberstring{\arabic{categorynumber}}%
\expandafter\dtbfillcategorytag\expandafter{\categorynumberstring}%
\expandafter\dtbfillsize\expandafter{\unit@size}%
\expandafter\dtbfilltype\expandafter{\unit@type}%
\expandafter\dtbfillscoring\expandafter{\unit@scoring}%
%
\expandafter\dtbfillglobalname\expandafter{\unit@globalname}%
\expandafter\dtbfillglobal@Ad\expandafter{\unit@global@Ad}%
\expandafter\dtbfillglobal@Ma\expandafter{\unit@global@Ma}%
\expandafter\dtbfillglobal@Adfly\expandafter{\unit@global@Adfly}%
\expandafter\dtbfillglobal@Mafly\expandafter{\unit@global@Mafly}%
\expandafter\dtbfillglobal@Di\expandafter{\unit@global@Di}%
%
\expandafter\dtbfillglobal@Rsr\expandafter{\unit@global@Rsr}%
\expandafter\ifblank\expandafter{\unit@global@Rsr}{}{%
	\cleannumberforsorting{\unit@global@Rsr}% output is saved to \costtwithoutformatting
	\expandafter\dtbfillglobal@RsrSortLabel\expandafter{\costwithoutformatting}%
}%
\expandafter\dtbfillglobalrules\expandafter{\unit@globalrules}%
%
\expandafter\dtbfillglobalnameB\expandafter{\unit@globalnameB}%
	\expandafter\dtbfillglobalB@Ad\expandafter{\unit@globalB@Ad}%
	\expandafter\dtbfillglobalB@Ma\expandafter{\unit@globalB@Ma}%
	\expandafter\dtbfillglobalB@Adfly\expandafter{\unit@globalB@Adfly}%
	\expandafter\dtbfillglobalB@Mafly\expandafter{\unit@globalB@Mafly}%
	\expandafter\dtbfillglobalB@Di\expandafter{\unit@globalB@Di}%
	%
	\expandafter\dtbfillglobalB@Rsr\expandafter{\unit@globalB@Rsr}%
	\expandafter\ifblank\expandafter{\unit@globalB@Rsr}{}{%
		\cleannumberforsorting{\unit@globalB@Rsr}% output is saved to \costtwithoutformatting
		\expandafter\dtbfillglobalB@RsrSortLabel\expandafter{\costwithoutformatting}%
	}%
	\expandafter\dtbfillglobalrulesB\expandafter{\unit@globalrulesB}%
%
\expandafter\dtbfillglobalnameC\expandafter{\unit@globalnameC}%
	\expandafter\dtbfillglobalC@Ad\expandafter{\unit@globalC@Ad}%
	\expandafter\dtbfillglobalC@Ma\expandafter{\unit@globalC@Ma}%
	\expandafter\dtbfillglobalC@Adfly\expandafter{\unit@globalC@Adfly}%
	\expandafter\dtbfillglobalC@Mafly\expandafter{\unit@globalC@Mafly}%
	\expandafter\dtbfillglobalC@Di\expandafter{\unit@globalC@Di}%
	%
	\expandafter\dtbfillglobalC@Rsr\expandafter{\unit@globalC@Rsr}%
	\expandafter\ifblank\expandafter{\unit@globalC@Rsr}{}{%
		\cleannumberforsorting{\unit@globalC@Rsr}% output is saved to \costtwithoutformatting
		\expandafter\dtbfillglobalC@RsrSortLabel\expandafter{\costwithoutformatting}%
	}%
	\expandafter\dtbfillglobalrulesC\expandafter{\unit@globalrulesC}%
%
\expandafter\dtbfilldefensename\expandafter{\unit@defensename}%
\expandafter\dtbfilldefense@HP\expandafter{\unit@defense@HP}%
\expandafter\dtbfilldefense@Df\expandafter{\unit@defense@Df}%
\expandafter\dtbfilldefense@Re\expandafter{\unit@defense@Re}%
\expandafter\dtbfilldefense@Arm\expandafter{\unit@defense@Arm}%
\expandafter\dtbfilldefense@Aeg\expandafter{\unit@defense@Aeg}%
\expandafter\dtbfillprintAeg\expandafter{\unit@printAeg}%
\expandafter\dtbfilldefenserules\expandafter{\unit@defenserules}%
\expandafter\dtbfilldefensearmour\expandafter{\unit@defensearmour}%
%
\expandafter\dtbfilldefensenameB\expandafter{\unit@defensenameB}%
	\expandafter\dtbfilldefenseB@HP\expandafter{\unit@defenseB@HP}%
	\expandafter\dtbfilldefenseB@Df\expandafter{\unit@defenseB@Df}%
	\expandafter\dtbfilldefenseB@Re\expandafter{\unit@defenseB@Re}%
	\expandafter\dtbfilldefenseB@Arm\expandafter{\unit@defenseB@Arm}%
	\expandafter\dtbfilldefenseB@Aeg\expandafter{\unit@defenseB@Aeg}%
	\expandafter\dtbfilldefenserulesB\expandafter{\unit@defenserulesB}%
	\expandafter\dtbfilldefensearmourB\expandafter{\unit@defensearmourB}%
%
\expandafter\dtbfilldefensenameC\expandafter{\unit@defensenameC}%
	\expandafter\dtbfilldefenseC@HP\expandafter{\unit@defenseC@HP}%
	\expandafter\dtbfilldefenseC@Df\expandafter{\unit@defenseC@Df}%
	\expandafter\dtbfilldefenseC@Re\expandafter{\unit@defenseC@Re}%
	\expandafter\dtbfilldefenseC@Arm\expandafter{\unit@defenseC@Arm}%
	\expandafter\dtbfilldefenseC@Aeg\expandafter{\unit@defenseC@Aeg}%
	\expandafter\dtbfilldefenserulesC\expandafter{\unit@defenserulesC}%
	\expandafter\dtbfilldefensearmourC\expandafter{\unit@defensearmourC}%
%
\expandafter\dtbfilloffensename\expandafter{\unit@offensename}%
\expandafter\dtbfilloffense@Ag\expandafter{\unit@offense@Ag}%
\expandafter\dtbfilloffense@At\expandafter{\unit@offense@At}%
\expandafter\dtbfilloffense@Of\expandafter{\unit@offense@Of}%
\expandafter\dtbfilloffense@St\expandafter{\unit@offense@St}%
\expandafter\dtbfilloffense@AP\expandafter{\unit@offense@AP}%
\expandafter\dtbfilloffenserules\expandafter{\unit@offenserules}%
\expandafter\dtbfilloffenseweapons\expandafter{\unit@offenseweapons}%
%
\expandafter\dtbfilloffensenameB\expandafter{\unit@offensenameB}%
	\expandafter\dtbfilloffenseB@Ag\expandafter{\unit@offenseB@Ag}%
	\expandafter\dtbfilloffenseB@At\expandafter{\unit@offenseB@At}%
	\expandafter\dtbfilloffenseB@Of\expandafter{\unit@offenseB@Of}%
	\expandafter\dtbfilloffenseB@St\expandafter{\unit@offenseB@St}%
	\expandafter\dtbfilloffenseB@AP\expandafter{\unit@offenseB@AP}%
	\expandafter\dtbfilloffenserulesB\expandafter{\unit@offenserulesB}%
	\expandafter\dtbfilloffenseweaponsB\expandafter{\unit@offenseweaponsB}%
%
\expandafter\dtbfilloffensenameC\expandafter{\unit@offensenameC}%
	\expandafter\dtbfilloffenseC@Ag\expandafter{\unit@offenseC@Ag}%
	\expandafter\dtbfilloffenseC@At\expandafter{\unit@offenseC@At}%
	\expandafter\dtbfilloffenseC@Of\expandafter{\unit@offenseC@Of}%
	\expandafter\dtbfilloffenseC@St\expandafter{\unit@offenseC@St}%
	\expandafter\dtbfilloffenseC@AP\expandafter{\unit@offenseC@AP}%
	\expandafter\dtbfilloffenserulesC\expandafter{\unit@offenserulesC}%
	\expandafter\dtbfilloffenseweaponsC\expandafter{\unit@offenseweaponsC}%
	\expandafter\dtbfilloffenseCcomment\expandafter{\unit@offenseCcomment}%
%
\expandafter\dtbfilloffensenameI\expandafter{\unit@offensenameI}%
	\expandafter\dtbfilloffenseI@Ag\expandafter{\unit@offenseI@Ag}%
	\expandafter\dtbfilloffenseI@St\expandafter{\unit@offenseI@St}%
	\expandafter\dtbfilloffenseI@AP\expandafter{\unit@offenseI@AP}%
	\expandafter\dtbfilloffenserulesI\expandafter{\unit@offenserulesI}%
	\expandafter\dtbfilloffenseweaponsI\expandafter{\unit@offenseweaponsI}%
% End of Unit Entry
\ifdefined\thisisanarmybook%
%
\else%
\def\@minipagerestore{}%
\fi%
\togglefalse{insideunitentry}%
}% END \iftoggle{dontdoanythingjustload}{
}% END newcommand unitentry

\newcommand{\unitheader}[1]{%
	\toggletrue{insideunitentry}%
	\def\@minipagerestore{\setlength{\parskip}{\mycurrentparskip}}% Else paragraph and title spaces behave differently when inside or outside a minipage - Unit entry spacing has been coded with this in mind, if you remove it, all spaces will be off
	\defunit{#1}%
	%
	\begin{minipage}[t]{\textwidth}% To avoid splitted profiles between 2 pages
	%
	%%%%%%%%%%%%%%%%%%%%%%
	% Unit title, logos and general info
	%
	\def\temphypertag{}%
	\ConvertNameMacroToStr{\unit@name}{\temphypertag} % remove first the \ to the left and then the {} to the right
	\xdef\unithypertag{\bookprefix\arabic{categorynumber}\temphypertag}% using the unit name macro to define an hypertag
	\hypertarget{\unithypertag}{}%
	\unitentryheader{}%
	% End of Unit Entry
	\ifdefined\thisisanarmybook%
	%
	\else%
	\def\@minipagerestore{}%
	\fi%
	\togglefalse{insideunitentry}%
	\vspace*{-3pt}%
	\hfill{\verysmallfontsize\textcolor{white}{d}} % Else footer goes wild for some reason with ocgcolorlinks option
	\end{minipage}%
	\vspace{-17.1pt}\par%
}
