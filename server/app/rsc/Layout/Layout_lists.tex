
%\setlist[itemize]{label=\textbullet}
\newcommand{\listsymbol}{\textbullet}

\ifdefined\thisistherulebook

	\setlist[itemize]{leftmargin=10pt}
	\setlist[enumerate]{leftmargin=13pt}
	
\fi

%%% Captions lists %%%

\newcommand{\captionlist}{\vspace*{3pt}\newline}
\newcommand{\captionitem}{\hspace*{5pt}}


%%% Lists with prices %%%

\newcommand{\startpricelist}{\raggedcolumns\begin{multicols}{2}\begin{description}[leftmargin=0.3cm, labelindent=0cm, labelsep=0.1cm, itemsep=8pt]}
\def\endpricelist{\end{description}\end{multicols}}
\newcommand{\pricelistitem}[3]{%
		% Should we add an English name after the standard name of the item?
		\def\tempENname{}%
		\ConvertNameMacroToStr{#1}{\tempENname}%
		\CombineStringsIntoMacroAndExpand{\tempENname}{EN}{\tempENname}%
        \expandafter\ifblank\expandafter{\tempENname}{% then there is nothing to add				
				\item \begin{samepage}\textbf{#1}\predotfill\hfill\nobreak\pointsblock{\pts{#2}}%
		}{% then we add the English name
			% Now see if it all fits on a single line
			\setbox0=\vbox{XX\textbf{#1} \textit{(\tempENname)}\predotfill\hfill\nobreak\pointsblock{\pts{#2}}}%
			\categorynoteheight=\ht0 \advance\categorynoteheight by \dp0\relax% Two lines do approx. a bit less than 20 pt
			\ifdim\categorynoteheight>15pt\relax% Name's too big we need to put it on two lines
					\item \begin{samepage}\textbf{#1}\predotfill\hfill\nobreak\pointsblock{\pts{#2}}\newline%
				\textit{(\tempENname)}%
			\else%
					\item \begin{samepage}\textbf{#1} \textit{(\tempENname)}\predotfill\hfill\nobreak\pointsblock{\pts{#2}}%
			\fi%
		}%
		\newline{}#3%
	\end{samepage}\par%
}
\newcommand{\nopricelistitem}[1]{\item \textbf{#1}\newline}

\newcommand{\startpricelistNSP}{\begin{description}[leftmargin=0.3cm, labelindent=0cm, labelsep=0.1cm, itemsep=8pt]}
\def\endpricelistNSP{\end{description}}



\newcommand{\startitemlist}{\raggedcolumns\begin{multicols}{2}\begin{description}[leftmargin=0.3cm, labelindent=0cm, labelsep=0.1cm, itemsep=8pt]}
\def\enditemlist{\end{description}\end{multicols}}
\newcommand{\listitem}[1]{\item[#1\spacebeforecolon{}:]}

\newcommand{\startitemlistonecol}{\begin{description}[leftmargin=0.3cm, labelindent=0cm, labelsep=0.1cm, itemsep=8pt]}
\def\enditemlistonecol{\end{description}}
\newcommand{\listitemonecol}[1]{\item \textbf{#1\spacebeforecolon{}:}\newline}

\newenvironment{smallitemize}{\begin{itemize}[itemsep=3pt,topsep=3pt]}{\end{itemize}}
\newenvironment{smallitemizeitem}{\begin{itemize}[noitemsep,topsep=-6pt]}{\end{itemize}}
\newenvironment{smallenumerate}{\begin{enumerate}[itemsep=3pt,topsep=3pt]}{\end{enumerate}}
\newenvironment{smallenumerateitem}{\begin{enumerate}[noitemsep,topsep=-6pt]}{\end{enumerate}}

%%% Army Model Rules sorting

\newcommand{\BigArmyModelRule}[3][]{%
% [optional] Rule type
% Name
% text
	\ConvertNameMacroToStr{#2}{\tempENname}%
	\CombineStringsIntoMacroAndExpand{\tempENname}{EN}{\tempENname}%
	\subsubtitle{%
		#2%
		\expandafter\ifblank\expandafter{\tempENname}{}{\textnormal{\largefontsize{} \textit{(\tempENname)}}}% 
		\expandafter\ifblank\expandafter{#1}{}{\textnormal{\largefontsize{} -- #1}}%
	}%
	#3\par%
}%


\newcommand{\startAMRsortedlist}{%
	\DTLifdbexists{AMRlist}{\DTLcleardb{AMRlist}}{\DTLnewdb{AMRlist}}% Create new/discard old list
}

\newcommand{\closeAMRsortedlist}[1][-]{%
% optional flag
% [-] - nothing [default]
% [m] - use minipage
	\DTLsort*{sortlabel}{AMRlist}% Sort list
	\DTLforeach*{AMRlist}{\name=name,\ruletext=ruletext,\AMRruletype=ruletype}{%
		\if m#1% 
			\begin{minipage}{\columnwidth}%
		\fi%
		\BigArmyModelRule[\AMRruletype]{\name}{\ruletext}%
		\if m#1% 
			\end{minipage}%
			\vspace{20pt}\newline%
		\fi%		
	}%
	\DTLcleardb{AMRlist}%
}%

\newcommand{\AMRsortedlistentry}[3][]{%
	\DTLnewrow{AMRlist}%
	\cleanstringforsorting{#2}% output is saved to \textwithoutformatting		
	\def\DTLentrycommand{\DTLnewdbentry{AMRlist}{sortlabel}}%
	\expandafter\DTLentrycommand\expandafter{\textwithoutformatting}%
	\DTLnewdbentry{AMRlist}{name}{#2}%
	\DTLnewdbentry{AMRlist}{ruletext}{#3}%
	\DTLnewdbentry{AMRlist}{ruletype}{#1}%
}



%%% Model rules %%%

% Model rule entry.
\newcommand{\modelruledef}[2]{%
	\DTLnewrow{unitruleslist}%
	\DTLnewdbentry{unitruleslist}{name}{#1}%
	\DTLnewdbentry{unitruleslist}{ruletext}{#2}%
	\cleanstringforsorting{#1}% output is saved to \textwithoutformatting		
	\def\DTLentrycommand{\DTLnewdbentry{unitruleslist}{sortlabel}}%
	\expandafter\DTLentrycommand\expandafter{\textwithoutformatting}%
}

\newcommand{\unitrulelist}[1]{%
	\DTLifdbexists{unitruleslist}{\DTLcleardb{unitruleslist}}{\DTLnewdb{unitruleslist}}% Create new/discard old list
	#1%
	\DTLifdbempty{unitruleslist}{}{%
		\DTLsort*{sortlabel}{unitruleslist}% Sort list
		\begin{description}[leftmargin=0.3cm, labelindent=0cm, labelsep=0.1cm, itemsep=0.15cm, parsep=0cm]%
			\DTLforeach*{unitruleslist}{\ruletext=ruletext,\nameoftherule=name}{% Print back the ordered list
				\item[\nameoftherule{}\spacebeforecolon{}:]\ruletext%
			}%
		\end{description}%
		\DTLcleardb{unitruleslist}%
	}
}
